\begin{apendicesenv}

\partapendices

% \textbf{ \\
%   Título:  \\
%   Autores: \\
%   Data:} \\

% \section{Introdução}
% 	Tópico que aborda sobre a teoria do experimento.

% \section{Materiais}
% 	\begin{itemize}
% 		\item Material 1
% 		\item Material 2
% 		\item Material n		
% 	\end{itemize}

% \section{Procedimento Experimental}
% 	Descrição de como foi realizado o experimento

% \section{Resultados e Discussão}
% 	Discussão a respeito dos resultados obtidos

% \section{Conclusão}
% 	Relatar se o experimento foi bem sucedido ou não.

\section{Atividades Desempenhadas no projeto}

\subsection{Software}

\subsubsection{Victor Henrique}

Contribuiu, desde o planejamento do projeto até a implementação do WebService, da aplicação mobile
e do sistema de validação de compras. Auxiliou no planejamento e execução dos testes do funcionamento
de cada subsistema, e integração com os demais subsistemas.

\subsubsection{Filipe Ribeiro}

A contribuição para com o projeto se deu principalmente no desenvolvimento da aplicação para dar inicio
a serventia de chopp. Auxilio no inicio do desenvolvimento do aplicativo e do WebService desde as partes
iniciais de fundamentação de ideias e requisitos. Houve uma contribuição para com o interfaceamento entre 
as aplicações embarcadas e aplicação de tiragem de chopp, além de manueseio de equipamentos para o acabamento
da estrutura.

\subsubsection{Edson Gomes}

\subsubsection{Phelipe Wener}

\subsection{Eletrônica}

\subsubsection{Lucas Raposo}

Contribuição foi o projeto, desenvolvimento e testes do circuito retificador
e inversor para carregar a bateria, contando com simulações, fabricação de PCB,
soldagem dos componentes. Outra contribuição foi o dimensionamento do transformador 
do retificador e do inversor e pesquisa de preço deles. Também foi projetada e fabricada
a PCB dos módulos dos motores de passo.Contribuição foi o projeto desenvolvimento
e testes do circuito retificador e inversor para carregar a bateria, contando com simulações, 
fabricação de PCB, soldagem dos componentes. Outra contribuição foi o dimensionamento
do transformador do retificador e do inversor e pesquisa de preço deles. Também foi projetada
e fabricada a PCB dos módulos dos motores de passo.

\subsubsection{Gabriel Henrique}

Construção dos códigos e implementação do sistema de controle de temperatura,
tiragem de chopp e liberação de copos, auxiliou no acabamento estrutural da máquina,
nos componentes e placas do Nobreak e na construção Chirller. Participou na construção
das placas impressas e configuração da Raspberry.

\subsubsection{Oziel Silva}

Colaborou no desenvolvimento de software embarcado para leitura de sensor. 
Participou na construção das placas de circuito impresso. 
Participou do desenvolvimento do sistema de tiragem de chope, 
auxiliou no acabamento estrutural da máquina. Auxiliou na montagem do chirller. 
Trabalhou nos testes de validação. Cedeu a casa pra reuniões e construções de módulos.

\subsubsection{Ithallo Guimarães}

Colaborou no desenvolvimento de software embarcado para leitura dos sensores , bem como no controle dos atuadores.  
Realizou a configuração da Raspberry pi. Atuou na construção das placas de circuito impresso. 
Participou do desenvolvimento do sistema de tiragem de chopp. 
Auxiliou na montagem do chirller. 
Trabalhou nos testes de validação do sistema.
Trabalhou de forma intensa na integração com os sistemas de software e energia, desenvolvendo códigos de controle.
Auxiliou a construção do nobreak, fazendo o código de controle do microcontrolador.
Atuou como tesoureiro do grupo.

\subsection{Energia}

\subsubsection{Clóves Júnior}

Colaborou no dimensionamento do sistema de potência do nobreak, dimensionamento do sistema de proteção e
do sistema de refrigeração. Implementação e testes dos subsistemas citados anteriormente e integração do
sistema de energia com os outros subsistemas.

\subsubsection{Jéssica Brito}

Auxílio no projeto do sistema de refrigeração,
auxílio na simulação do projeto de refrigeração. 
Dimensionamento dos componentes, compra dos componentes e montagem do sistema. 
Auxílio na retiragem do compressor do sistema anterior. 
Auxílio na montagem do sistema de proteção, cooperação na montagem do nobreak.
Auxílio na integração com a estrutura e a eletrônica.

\subsubsection{Gabriela Volpato}

Projeto do sistema de refrigeração onde atividades quanto auxilio na manufatura do Chirller.
Auxilio na montagem do sistema de refrigeração e a inserção de gás. 
Asseguramento das entradas e saídas do sistema assim como as portas de integração com as 
demais áreas do projeto. Cooperação na aquisição de equipamentos do sistema de proteção 
elétrico e auxilio nos testes referentes ao NoBreak.

\subsection{Estrutura}

\subsubsection{Guilherme Matias}

\subsubsection{Felipe Côrrea}

\noindent

\end{apendicesenv}
