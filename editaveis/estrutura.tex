\chapter[Estrutura]{Estrutura}

\section{Requisitos do Projeto}


\begin{itemize}

\item{Dimensões: Devem estar adequadas às necessidades de alocar elementos e componentes que compõem a chopeira permitindo seu correto funcionamento. Devem ser avaliadas questões como ergonomia do usuário, acessibilidade e facilidade de manutenção (todas essas relacionadas diretamente às dimensões), através de dimensões que permitam fácil acesso ao mantenedor.}

\item{Material: O material deve estar em condições de suportar variações de temperatura impostas pelo funcionamento do sistema de refrigeração e também favorecer e facilitar os processos de fabricação. O design e aparência são itens diretamente influenciáveis na entrega final da estrutura}

\item{Comportamento Estrutural e Características: A estrutura deve ser leve, o que facilita o transporte e manuseio, mantendo boa resistência, suportando vibrações, cargas e solicitações de fadiga}

\end{itemize}

\section{Projeto}

Após a realização do packaging, começaram os trabalhos de integração com os outros subsistemas para levantamento de requisitos. Após algumas reuniões foi verificado que a estrutura deveria ser capaz de suportar o peso de dois barris de chopp de 67,4Kg cada e dois cilindros de CO2 de 20kg no primeiro nível. O segundo nível suportaria todo o sistema de refrigeração e alimentação do circuito elétrico da máquina que pesaria no total em torno de 50Kg e o terceiro nível seria o reservatório dos copos. Entre o segundo e terceiro níveis existem dois subníveis que acomodam toda parte eletrônica e de automação da máquina, como haspberrys, um motor de passo e algumas placas de circuito.

Foi decidido que o primeiro nível não seria feito com chapa de alumínio como decidido anteriormente mas sim, com aço. Houve a opção de apenas aplicar uma treliça a este nível e usar uma chapa de alumínio mais espessa porém o custo ficou muito elevado. A solução final foi aplicar treliças no primeiro e segundo níveis e usar chapas mais finas (2mm ante 4mm que era usado no projeto inicial). Esta solução foi a que gerou a melhor relação entre peso, custo e resistência mecânica (A deflexão máxima passou a ser de aproximadamente 2mm).

Todo o acabamento final, que é responsável por toda a estética do equipamento, foi feito com chapas de alumínio cortadas e dobradas. É importante ressaltar que a equipe de estrutura ficou responsável pelo mecanismo de acionamento da base que inclina o copo. Foi projetado um mecanismo simples de biela-manivela fabricado através de impressão 3d. A manivela é rotacionada por um motor de passo que empurra a biela contra a base de forma que  pivota no suporte superior.

\section{Solução}

As soluções apresentadas foram realizadas levando-se em consideração aplicações semelhantes existentes no mercado para os elementos em questão. No caso da porta existente na máquina para acesso aos componentes internos para manutenção, a utilização de dobradiça se mostrou a mais assertiva, como pode ser visto em foto.


A utilização de chapas, na busca da redução de peso e boa resistência mecânica para acoplamento de sistemas e elementos também foi uma solução utilizada. A possibilidade de bom acabamento com tais chapas foram levadas em consideração, para tanto isso refletiu em uma necessidade de cortes internos para a chapa frontal, realizados em corte a plasma. 

Concomitantemente a isso, a utilização de folhas de alumínio em algumas partes para redução de custos e peso também foi uma decisão da estrutura. Foram fixadas com fitas 3m de alta resistência e rebite, diminuindo a necessidade da utilização de parafusos. Já que nesse caso não há a necessidade de frequente montagem e desmontagem.


Algumas soluções de pintura e acabamento com uso de borrachas de vedação e diminuição de ruído interno foram aplicadas. Todos os pontos aqui citados são complementares as soluções de fabricação e montagem já discutidas no relatório existente no ponto de controle 2.