\chapter[Energia]{Energia}
    \section[NoBreak]{NoBreak}
        \subsection[Requisitos do Projeto]{Requisitos do Projeto}
            \subsubsection[Retificador]{Retificador}
            	\textbf{Dado} que a concessionária de energia forneça tensão
            	
            	\textbf{E} a máquina esteja em funcionamento
            	
			    \textbf{Então} a tensão da bateria deve ser carregada 
			    
            \subsubsection[Inversor]{Inversor}
            	\textbf{Dado} que a bateria forneça tensão
            	
            	\textbf{E} a máquina esteja em funcionamento
            	
			    \textbf{Então} a maquina deve ter tensão de alimentação
			    
        \subsection[Projeto]{Projeto}     
            \subsubsection[Retificador]{Retificador}            
				O projeto do retificador do NoBreak tem o objetivo de carregar a bateria do 					dispositivo enquanto houver energia fornecida da concessionária. O retificador 				deve ser um circuito estável, mantendo o padrão de fornecer sempre a tensão de 				saída necessária para a carga (Bateria). Logo, os requisitos desse subproduto 					são:

                \begin{itemize}
                	\item Tensão de entrada: $220V_{AC}$
                	\item Tensão de saída: $12V_{DC}$
                	\item Corrente mínima: $5A$
                \end{itemize}
		
				Atendendo a esses requisitou a proposta do projeto é um retificador linear com 				a corrente máxima fornecida a carga de 5A seguindo os seguintes passos:
				
                \begin{figure}[!htb]
            		\centering
            		\includegraphics[scale= 0.5]{figuras/Diagrama_Retificador.png}
            		\caption{Diagrama de blocos do Retificador. Fonte: Própria.}
            		\label{diagrama-retificador}
            	\end{figure}
            	
				Para solucionar o que proposto, o primeiro passo foi calcular os parâmetros do 				circuito para suportar a corrente de 5A requisitada. A tensão do primário deve 				ser a tensão fornecida pela concessionária, no caso 220V. Para a tensão do 						secundário, deve-se utilizar a tensão de saída do circuito retificador, porém 					sabendo que o bloco retificador da figura 1 causa uma queda de tensão pequena, 				utilizou-se a tensão de secundário acima da necessária (12V). O valor 							comercial acima mais próximo para tensão de saída é de 16v. Para o 								transformador, necessitou-se o cálculo da potência dele.

          	  	\begin{equation}
                	S = V*I = 16*5 = 80VA
            	\end{equation}
            	
				Onde
				
				V = Tensão no secundário do transformador
				
				I = Corrente requisitada pelo projeto
				
				S = Potência complexa (Volt-Ampère)

				O transformador escolhido foi comprado pela internet com as exatas 								características propostas.

                \begin{figure}[!htb]
            		\centering
            		\includegraphics[scale= 0.3]{figuras/Transformador.png}
            		\caption{Transformador do Retificador. Fonte: Própria.}
            		\label{transformador-retificador}
            	\end{figure}	
            	
            	O bloco Retificador escolheu-se um retificador de meia ponte devido ao 							transformador ter um enrolamento secundário com Tap central. O funcionamento 					deste bloco é ceifar o lado negativo da tensão alternada. 			

                \begin{figure}[!htb]
            		\centering
            		\includegraphics[scale= 0.5]{figuras/Retificador.png}
            		\caption{Gráfico Tensão de entrada e saída de um diodo. Fonte: Própria.}
            		\label{retificador}
            	\end{figure}		
            	
            	Como são duas bobinas invertidas no secundário do transformador, uma bobina 					vai ceifar o lado negativo da tensão e a outra o lado positivo da tensão, ao 					somarem os dois resultados, o circuito resultante irá realizar o mesmo papel 					de um retificador de onda completa.
            	
				O diodo retificador deve suportar a corrente máxima do projeto de 5A. Para 						isso, utilizou-se um diodo 6A10 que de acordo com o datasheet, a corrente 						máxima suportada é de 5A.
				
				Observa-se que a tensão resultante após o bloco retificador oscilará entre um 					pulso e outro de forma relevante. Essa oscilação denomina-se Ripple. Para 						deixar a tensão de saída mais linear possível, utiliza-se o bloco redutro de 					ripple. Este bloco  tem como função atrasar o tempo de carregamento da onda, 					fazendo com que no momento do descarregamento a saída deste bloco não consiga 					acompanhar, mantendo uma tensão mais estável. Para isso, utiliza-se um 							capacitor em paralelo com o circuito. O capacitor deve ser grande para atrasar 				a onda, logo utiliza-se capacitores eletrolíticos, e suportar a tensão de pico 				da onda (No caso no mínimo 16v) Para isso utilizou-se um capacitor de 1000uF e 				50v.

                \begin{figure}[!htb]
            		\centering
            		\includegraphics[scale= 0.2]{figuras/Capacitor.jpg}
            		\caption{Capacitor redutor de Ripple. Fonte: Própria.}
            		\label{capacitor}
            	\end{figure}
            	
            	O bloco regulador de tensão tem a função de estabilizar a tensão para ficar o 					mais linear possível, porque para circuitos mais sensíveis a osicilação por 					menor que seja após o capacitor  pode ser prejudicial. Para realizar esta 						regulação, utiliza-se um diodo zener com a tensão zener exatamente na desejada 				para saída do projeto (No caso, 12v).  Para isso, utilizou-se um diodo zener 					4712A, cujo tensão zener é de 12v e a potência máxima é de 1w. Para proteção 					contra correntes invertida no circuito vinda da bateria, utilizou-se um diodo 					retificador 6A10 na saída do circuito no sentido da corrente convencional.
		
				O circuito completo do retificador foi simulado pelo software proteus e 						comprovado:
				
                \begin{figure}[!htb]
            		\centering
            		\includegraphics[scale= 0.4]{figuras/Circuito_Retificador.png}
            		\caption{Simulação Retificador de tensão 12v. Fonte: Própria.}
            		\label{retificador-completo}
            	\end{figure}
 
             \subsubsection[Inversor]{Inversor}
				O projeto do inversor do NoBreak tem o objetivo de transformar a tensão 						contínua contida na bateria e transformá-la em tensão alternada para alimentar 				a carga . O inversor deve ser um circuito estável, mantendo o padrão de 						fornecer sempre a tensão de saída necessária para a carga (Máquina de Chopp). 					Logo, os requisitos desse subproduto são:
				
                \begin{itemize}
                	\item Tensão de entrada: $12V_{DC}$
                	\item Tensão de saída: $220V_{AC}$
                	\item Corrente mínima: $3A$
                \end{itemize}
                
                Atendendo a esses requisitou a proposta do projeto é um inversor com a 							corrente máxima fornecida a carga de 4,5A seguindo os seguintes passos:
                
                \begin{figure}[!htb]
            		\centering
            		\includegraphics[scale= 0.4]{figuras/Diagrama_Inversor.png}
            		\caption{Diagrama de blocos do Inversor. Fonte: Própria.}
            		\label{diagrama-inversor}
            	\end{figure} 
            	Para solucionar o que proposto, o primeiro passo foi calcular os parâmetros do 				circuito para suportar a corrente de 4,5A requisitada. A tensão do primário 					deve ser a tensão da bateria, no caso 12v. Para a tensão do secundário, deve-					se utilizar a tensão que a carga necessita, neste caso 220Vac,. Para o 							transformador, necessitou-se o cálculo da potência dele.    
            	
            	\begin{equation}
                	S = V*I = 220*4,5 = 1kVA
            	\end{equation}
                
                Onde	
                
                V = Tensão no secundário do transformador
                
				I = Corrente requisitada pelo projeto
	
				S = Potência complexa (Volt-Ampère)
				
				O transformador escolhido foi fabricado com as exatas características 							propostas.		

                \begin{figure}[!htb]
            		\centering
            		\includegraphics[scale= 0.3]{figuras/Transformador_Inversor.jpg}
            		\caption{Transformador do inversor. Fonte: Própria.}
            		\label{transformador-inversor}
            	\end{figure}  
     
            	O bloco oscilador tem como função criar uma onda modelo para criar a tensão 					alternada. Essa onda modelo deve ter frequência de 60Hz assim como é na tensão 				utilizada pela concessionária de energia. O primeiro teste de oscilador foi 					com um circuito oscilador a partir de um CI 4047. Esse CI tem a função de 						alternar entre uma saída e outra uma tensão na frequência determinada por uma 					equivalência entre resistor e capacitor na entrada do CI. Ao fazerem testes 					observou-se que a tensão de saída em uma das portas oscilava com a frequência 					duas vezes maior que a frequência na outra porta. A conclusão tirada desses 					testes é que o CI estava com defeito, porém por falta de tempo, adotou-se 						outra solução mais rápida.
            	
				A segunda solução aplicada foi realizar um circuito oscilado a partir de um 					microcontrolador. O controlador escolhido foi um AtTiny por ser pequeno, 						econômico e rápido o suficiente para esta aplicação. O programa compilado nele 				faz a função de escrever nível lógico alto em uma porta, aguardar 16,6ms 						(Aproximadamente 60Hz) escrever nível lógico baixo nesta porta e o inverso com 				outra porta. 	

                \begin{figure}[!htb]
            		\centering
            		\includegraphics[scale= 0.2]{figuras/Attiny.jpg}
            		\caption{Attiny 85. Fonte: Própria.}
            		\label{attiny}
            	\end{figure} 	
            	
            	A saída da etapa do oscilador é utilizada pelo bloco de ponte de MosFet. A 						ponte de MosFet tem a função de amplificar a onda gerada pelo oscilador além 					de ser capaz de suportar quantidades grandes de corrente. O MosFet escolhido 					para este projeto é o IRF2807, de acordo com o datasheet, o componente é capaz 				de suportar corrente de dreno de até 75A.	
            	
                \begin{figure}[!htb]
            		\centering
            		\includegraphics[scale= 0.2]{figuras/IRF2807.jpg}
            		\caption{MosFet IRF2807. Fonte: Própria.}
            		\label{mosfet}
            	\end{figure}            						
				
				Após a ponte de MosFet o circuito passa pelo transformador e filtro RLC para 					alcançar o mais próximo de uma senóide com tensão RMS de 220v. O filtro RLC 					foi projetado para o circuito abaixo:	

                \begin{figure}[!htb]
            		\centering
            		\includegraphics[scale= 1.0]{figuras/Filtro_RLC.png}
            		\caption{Filtro RLC. Fonte: Própria.}
            		\label{rlc}
            	\end{figure}  				
				
				O filtro apesar de ser RLC, conta apenas com indutor e capacitor, pois a 						resistência do circuito é a própria resistência do indutor.
				
				O circuito todo projetado foi simulado de acordo com a imagem abaixo:
				
                \begin{figure}[!htb]
            		\centering
            		\includegraphics[scale= 0.4]{figuras/Circuito_inversor.png}
            		\caption{Circuito inversor completo. Fonte: Própria.}
            		\label{inversor}
            	\end{figure} 
            					
				Alguns componentes como mostra a figura acima estão diferentes dos projetados 					pois no simulador não havia igual, mas foi pego o componente mais próximo para 				ser o mais próximo do real.          
             
        \subsection[Solução Adotada]{Solução Adotada}
            \subsubsection[Retificador]{Retificador} 
            	A partir dos testes de simulação montou-se o circuito em placa de fenolite. O 					design foi feito no software Traxmakker e para fabricação fez o processo 						térmico e depois corrosão.	
            	
                \begin{figure}[!htb]
            		\centering
            		\includegraphics[scale= 0.4]{figuras/Fenolite_Retificador.png}
            		\caption{Trilhas do circuito retificador. Fonte: Própria.}
            		\label{retificador-fenolite}
            	\end{figure}	
            	
                \begin{figure}[!htb]
            		\centering
            		\includegraphics[scale= 0.4]{figuras/Circuito_Retificador_Completo.png}
            		\caption{Circuito Retificador Completo. Fonte: Própria.}
            		\label{retificador-completo2}
            	\end{figure}	
            	
				Para realizar a integração, o circuito retificador foi posicionado no 							local apresentado pela equipe de estrutura para o Nobreak. Na entrada do 						transformador adaptou-se a partir de um conector múltiplo um cabo PP $2,5mm^2$ 				de duas vias com um conector de tomada macho padrão ABNT 14136.             	

                \begin{figure}[!htb]
            		\centering
            		\includegraphics[scale= 0.4]{figuras/Pardrao_tomada.jpg}
            		\caption{Tomada macho padrão ABNT. Fonte: Própria.}
            		\label{tomada}
            	\end{figure}	 
            	
				Para saída do retificador adaptadou-se um cabo $6mm^2$ com um terminal de 						bateria automotiva universal para poder ter fácil acesso de tirar e 							colocar a bateria em caso de transporte ou manutenção.  
				
                \begin{figure}[!htb]
            		\centering
            		\includegraphics[scale= 0.4]{figuras/Terminal_Bateria.jpg}
            		\caption{Terminal universal Bateria. Fonte: Própria.}
            		\label{terminal}
            	\end{figure}

            	Notou-se que a corrente do circuito feito em testes estava limitada, e o 						componente que realiza este efeito é a resistência em série do circuito. Com 					isso, foi decidido em diminuir o valor resistivo o suficiente para deixar 						próximo ao máximo de potência dissipada. O valor do resistor que chegou 						próximo ao desejado foi $12\Omega$ e $10W$. A corrente máxima que o resistor 					passará é:
            	
          	  	\begin{equation}
                	I = \frac{P}{V} = \frac{10}{12} = 0,833A
            	\end{equation}            	
            	
            	A corrente máxima de 830mA é o suficiente para o projeto visto que a bateria 					não estará sobre carga a todo momento.

             \subsubsection[Inversor]{Inversor}            
				Para a fabricação da placa me fenolite atentou-se aos pontos em que a corrente 				será alta fazendo trilhas largas com capacidade para suportar a corrente 						necessária.

                \begin{figure}[!htb]
            		\centering
            		\includegraphics[scale= 0.4]{figuras/Placa_inversor.png}
            		\caption{Placa impressa do inversor. Fonte: Própria.}
            		\label{inversor-projeto}
            	\end{figure} 		

				Percebe-se pelo circuito impresso que foram utilizados 5 mosfets para cada 						canal em paralelo. Colocar os Mosfets em paralelo faz com que a corrente por 					cada componente seja menor, aumentando a vida útil do componente e evitando 					que esquente excessivamente o sistema. Para cada canal de Mosfet será 							utilizado um dissipador de calor para melhorar a troca de calor do componente 					com o ambiente.		
		
                \begin{figure}[!htb]
            		\centering
            		\includegraphics[scale= 0.4]{figuras/Inversor_pronto.jpg}
            		\caption{Circuito inversor no circuito impresso. Fonte: Própria.}
            		\label{inversor-pronto}
            	\end{figure} 		
            	
				De acordo com a figura acima, a saída do oscilador está em amarelo tem tensão 					de saída de 3,4v e frequência de 59,82Hz. O mesmo canal amplificado pelo 						MosFet apresenta a mesma frequência porém com tensão de 12,6v. 
				
				Para evitar a queima dos Mosfets, o oscilador não pode ter em nenhum momento 					os dois canais ligados, pois isso gera um curto circuito, queimando assim os 					mosfets. Para resolver este problema utilizou-se de um recurso chamado Tempo 					morto. Este tempo morto é um tempo no qual as duas portas do oscilador ficam 					destivadas para não haver curto entre os mosfets. Este tempo foi programado no 				Attiny e será ajustado quando houver integralização entre os projetos.           
				
                \begin{figure}[!htb]
            		\centering
            		\includegraphics[scale= 0.2]{figuras/Tempo_morto.jpg}
            		\caption{Tempo morto em testes. Fonte: Própria.}
            		\label{tempo-morto}
            	\end{figure}
       

        \subsection[Casos de Teste]{Casos de Teste}
            \subsubsection[Retificador]{Retificador}       
           	Os primeiros testes mostraram que o circuito funcionou, porém a corrente 						máxima que o retificador forneceu foi cerca de $90mA$. Com essa corrente o 						Nobreak teria dificuldade para carregar a bateria, não sendo eficiente em caso 				de falta de energia da concessionária. 
            	
                \begin{figure}[!htb]
            		\centering
            		\includegraphics[scale= 0.6]{figuras/corrente_retificador.jpeg}
            		\caption{Corrente primeiro teste no retificador. Fonte: Própria.}
            		\label{corrente-retificador}
            	\end{figure}            	
             
            \subsubsection[Inversor]{Inversor}
            
            	A partir do circuito montado a saída do bloco da ponte de Mosfet em testes foi:
            	
                \begin{figure}[!htb]
            		\centering
            		\includegraphics[scale= 0.2]{figuras/saida_inversor.jpg}
            		\caption{Saída do bloco oscilador e ponte de MosFet. Fonte: Própria.}
            		\label{saida-inversor}
            	\end{figure}    
    
 
 
          	
             	
            	
            	
            	         
            	
 	
            	

				
				 	         		
		
        \section[Sistema de Refrigeração]{Sistema de Refrigeração}

            Na parte de planejamento do sistema de refrigeração, foram feitos todos os
            cálculos referentes ao sistema. 

            O dimensionamento da potência do compressor, necessário no ciclo de
            refrigeração foi feito a partir da Primeira Lei da Termodinâmica, onde calculou-se a
            carga térmica necessária a ser retirada do chopp, considerando que a vazão
            necessária a ser retirada da chopeira é 45l/h, ou aproximadamente, 0,00125m$^3$/s, a
            Primeira Lei para o compressor será:
            
            \begin{equation}
                Q_{(ponto)} = m_{(ponto)} \times (h_1 - h_2)
            \end{equation}

            Onde:
            \begin{itemize}
                \item Q = Quantidade de calor
                \item m = fluxo mássico (vazão multiplicada pela densidade)
                \item h = entalpia
            \end{itemize}
            
            Manipulando, têm-se:

            \begin{equation}
                h_1 - h_2 = Cp_{méd} \times (T_1 - T_2)
            \end{equation}

            Onde:
            \begin{itemize}
                \item $Cp_{médi}$ = Calor Específico a pressão constante
                \item T = Temperatura do fluido
            \end{itemize}
            então, substituindo na primeira equação teremos:

            \begin{equation}
                Q_{ponto} = m_{ponto} \times Cp_{méd} \times (T_1 - T_2)
            \end{equation}

            Consideramos como fluido para os cálculos a água, visto que o chopp tem bastante 
            água na sua composição e que não foram encontrados dados para o calor específico 
            do chopp, a aproximação é aceitável. O valor de calor específico a pressão constante
            encontrado para a água é de 4,18KJ/Kg. A temperatura considerada na entrada do 
            trocador de calor é a mesma que a temperatura ambiente de aproximadamente 25$^\circ$C,
            e a requerida na saída dele para estar de acordo com os requisitos é de
            aproximadamente 1$^\circ$C. Dessa forma, o fluxo mássico:

            \begin{equation}
                m_{ponto} = p \times V
            \end{equation}

            \begin{equation}
                m_{ponto} = 1000 \times 0,00125
            \end{equation}

            \begin{equation}
                m_{ponto} = 0,0125
            \end{equation}

            Logo, a carga térmica a ser retirada do fluido, no trocador de calor,
            pelo gás que sai do compressor será:
            
            \begin{equation}
                Q_{ponto} = 0.0125 \times 4.18 \times (25-1(-1)) = 1.35 KW
            \end{equation}

            Ou então:
            \begin{equation}
                46.32.48 Btu/h
            \end{equation}

            Analisando o catálogo (nome), para essa carga térmica seria necessário um compressor de
            $1 \over 2$ Hp.

            \subsubsection[Simulação do sistema de Refrigeração]{Simulação do sistema de Refrigeração}
                Para a segurança do funcionamento do sistema de refrigeração usou-se o
                Software CoolPack para a simulação do sistema como um todo. Nesse
                software dados construtivos físicos dos seguintes elementos foram inseridos: 
                Evaporador, Condensador, moto-compressor e tubulação. 

                A figura \ref{simulacao-refrigeracao} ilustra a simulação realizada no software juntamente com os os
                dados das temperaturas em cada estado. Essas temperaturas serão importantes para o
                devido entendimento da troca de calor entre evaporador e o líquido externo ao
                sistema a ser refrigerado.

                \begin{figure}[!htb]
            		\centering
            		\includegraphics[scale= 0.3]{figuras/simulacao-refrigeracao.png}
            		\caption{Simulação do sistema de Refrigeração. Fonte: Própria.}
            		\label{simulacao-refrigeracao}
            	\end{figure}

                Observa-se na imagem acima que foi inserido o dado correto quanto ao tipo de
                gás em que o compressor opera. Tem-se como resultado importante a quantidade
                de energia, ou seja, calor transferida no evaporador que corresponde a 4.00 kW.
                Esse dado será validado nos testes do sistema já montado

            %\subsection[Simulação Chiller]{Simulação Chiller}

            \subsection[Montagem e Construção do Sistema de Refrigeração]{Montagem e Construção do Sistema de Refrigeração}
                A montagem foi feita com apenas um chirller no protótipo por motivos de custo,
                sendo assim, a potência do compressor foi reduzida pela metade, ou seja, $1 \over 4$ Hp.
                O compressor foi cedido pelo professor Rander, para viabilizar que o protótipo
                fosse fabricado, as especificações desse compressor são as seguintes:

                \begin{figure}[!htb]
            		\centering
            		\includegraphics[scale= 0.3]{figuras/montagem-chiler.png}
            		\caption{Processo de Montagem do Sistema de Refrigeração. Fonte: Própria.}
            		\label{simulacao-refrigeracao}
            	\end{figure}
                
                \subsubsection[Descrição das etapas]{Descrição das etapas}
                    Para a montagem do sistema de refrigeração teve que seguir as seguintes etapas:                    
                    \subsubsubsection[Manufaturação do Chirller]{Manufaturação do Chirller}
                        Neste processo, os materiais utilizados foram: uma panqueca de alumínio $3 \over 8$ e
                        15 metros, uma mangueira atóxica trançada $3 \over 4$, 2 kits de engate rápido para
                        mangueira, 2  reduções de $3 \over 4$ para $1 \over 2$, 2 tês de PVC com extremidades de $3 \over 4$
                        e central de $1 \over 2$, 2 espigões de mangueira de $3 \over 4$ e 4 niples de união de $1 \over 2$,
                        2 braçadeiras de $3 \over 4$ e 2 adaptadores de gás de $1 \over 2$ para $3 \over 8$.
                        Primeiro, lavou-se a mangueira com água e sabão, feito isso,
                        desenrolou-se a panqueca em todo o seu comprimento, de forma que ela
                        ficasse bem reta. 
                        
                        Na segunda parte do processo, o tubo de alumínio foi inserido dentro da
                        mangueira, ainda de forma que que a estrutura ficasse reta. Depois,
                        enrolou-se a estrutura no molde, que tem 30cm de diâmetro, prendendo
                        ele com lacres de plástico. A imagem a seguir ilustra esse processo,
                        feito por membros do grupo:

                        \begin{figure}[!htb]
                            \centering
                            \includegraphics[scale= 0.3]{figuras/moldagem-chiller.png}
                            \caption{Moldagem do chirller. Fonte: Própria.}
                            \label{moldagem-chiller}
                        \end{figure}

                        Foi feito um desenho esquemático do Chirller no software CatiaV5 para
                        possibilitar as simulações, ele está ilustrado na figura \ref{desenho-chiller}.

                        \begin{figure}[!htb]
                            \centering
                            \includegraphics[scale= 0.3]{figuras/desenho-chiller.png}
                            \caption{Desenho esquemático do chirller. Fonte: Própria.}
                            \label{desenho-chiller}
                        \end{figure}

                        \begin{figure}[!htb]
                            \centering
                            \includegraphics[scale= 0.3]{figuras/corte-mangeira.png}
                            \caption{Vista em corte da mangueira com o tubo de alumínio inserido. Fonte: Própria.}
                            \label{vista-mangueira}
                        \end{figure}

                        Na terceira parte da passagem, passou-se fita veda-rosca em todos os acessórios de
                        tubulação, para evitar vazamentos, e montou-se o tê adaptado que possibilita a
                        entrada e saída de chopp no Chirller de contra-fluxo. Neste processo,
                        coloca-se o espigão em uma extremidade de $3 \over 4$ do tê, um nipple de união
                        de $1 \over 2$ na extremidade de $1 \over 2$ do tê, um engate rápido no nipple. Na outra
                        extremidade de $3 \over 4$, coloca-se uma redução de $3 \over 4$ pra $1 \over 2$, coloca-se
                        um nipple de união e por fim, coloca-se um adaptador de gás no nipple. Depois
                        de montado o tê adaptado ficou da seguinte maneira:

                        \begin{figure}[!htb]
                            \centering
                            \includegraphics[scale= 0.3]{figuras/entrada-saida-chopp.png}
                            \caption{Tê adaptado para entrada e saída de chopp. Fonte: Própria.}
                            \label{entrada-saida-chopp}
                        \end{figure}

                    \subsubsubsection[Desmontagem do Sistema]{Desmontagem do Sistema}
                        Nessa etapa teve-se o devido cuidado para a desmontagem do sistema. Retirou-se o gás
                        R22 com o equipamento próprio para tal ação, considerando assim a lei ambiental
                        quanto a emissão de gases poluentes na atmosfera. Dessa forma, na válvula
                        de serviço do compressor a válvula de descarga do fluido refrigerante
                        encaminhando-o para o devido recipiente de armazenamento de gás a vacuo.
                        Desse modo realizou-se o procedimento de recolhimento passivo. Esse procedimento e
                        indicado para quantidades pequenas de gas. Ele pode ser retirado na sua forma gasosa
                        ou líquida. Isso acontece devido a diferença de pressão entre os dois sistemas.

                        Esse procedimento foi realizado a fim de atender a Resolução CONAMA no 267, de 14 de
                        setembro de 2000. Essa resolução proíbe a utilização e consequentemente emissão
                        de substâncias que destroem a camada de ozônio. 

                    \subsubsubsection[Montagem do Sistema]{Montagem do Sistema}
                        Após a devida manufatura do Chiller e a desmontagem do sistema escolheu-se uma
                        base e alocou-se os componentes de acordo com a figura \ref{alocacao-dispositivos} 

                        \begin{figure}[!htb]
                            \centering
                            \includegraphics[scale= 0.2]{figuras/alocacao-dispositivos.png}
                            \caption{Sistema aberto e alocação dos dispositivos. Fonte: Própria.}
                            \label{alocacao-dispositivos}
                        \end{figure}

                        A base para o sistema consiste em uma de material madeira. Esse material foi
                        escolhido devido às suas propriedades térmicas e pela fácil disponibilidade desse. 
                        
                        Realizou-se testes de pressão no motocompressor e constatou-se que esse encontrava-se
                        com baixa capacidade de compressão. Sendo assim realizou-se a limpeza do motor e
                        também a troca de do filtro do sistema para o filtro secador do tipo Darfur.
                        Este filtro impede com que partículas indesejadas passem pela a tubulação e cheguem
                        no motocompressor. Além disso, como o filtro secador foi trocado optou-se, por medidas
                        preventivas, trocar o filtro capilar. 

                        A tubulação do Chiller escolhido consiste de alumínio e as demais tubulações do
                        sistema sao de cobre. Assim sendo, tornando inviável a solda desses dispositivos como
                        solução. A partir disso, optou-se pela aquisição da junção de gás (nipel)
                        com a cabeça em forma de cilindro ilustrado na figura \ref{nipel-gas}.
                        
                        \begin{figure}[!htb]
                            \centering
                            \includegraphics[scale= 0.2]{figuras/nipel-gas.png}
                            \caption{Nipel de gás. Fonte: Própria.}
                            \label{nipel-gas}
                        \end{figure}

                        Assim com a peça ilustrada acima realizou-se o processo de flangeamento dos tubos. Esse
                        processo consiste em alargar a espessura do tubo de cobre para que o tubo de alumínio
                        seja inserido e juntado pela peça acima não havendo vazamento de gás. O processo de
                        flangeamento e ilustrado na figura \ref{processo-flageamento}. 

                        \begin{figure}[!htb]
                            \centering
                            \includegraphics[scale= 0.2]{figuras/tubos-desconectados.png}
                            \caption{Tubos Desconectados. Fonte: Própria.}
                            \label{tubos-desconectados}
                        \end{figure}

                        \begin{figure}[!htb]
                            \centering
                            \includegraphics[scale= 0.2]{figuras/processo-flagelamento.png}
                            \caption{Processo de Flangeamento. Fonte: Própria.}
                            \label{processo-flageamento}
                        \end{figure}

                        Para a continuação da montagem do sistema, realizou-se a solda do tubo capilar
                        e filtro secador. Assim, com o sistema todo soldado e montado foi feita a
                        recarga de gás. Antes de abastecer o sistema, ligou-se o motocompressor para a
                        realização de uma câmara de vácuo dentro das tubulações. Depois dessa verificação
                        conectou-se o manifold nos diferentes pontos do sistema, parte de baixa e alta
                        pressão e conectou-se a mangueira de abastecimento permitindo a passagem de
                        fluido refrigerante. A carga foi realizada até que o sistema não suportasse
                        mais gás dentro desse. A figura \ref{recarga-gas} mostra a recarga do sistema com o
                        fluido refrigerante R22. 

                        \begin{figure}[!htb]
                            \centering
                            \includegraphics[scale= 0.2]{figuras/recarga-gas.png}
                            \caption{Recarga do sistema com gás R22. Fonte: Própria.}
                            \label{recarga-gas}
                        \end{figure}

                        Realizou-se os testes e notou-se que o condensador aplicado no sistema estava muito
                        pequeno para o evaporador escolhido. Assim sendo, repetiu-se todos os passos
                        acima quanto a soldagem dos tubos e a recarga de gás. Tendo assim o sistema de
                        refrigeração abaixo. 

                        Com o sistema pronto realizou-se testes e a partir desses montou-se a seguinte tabela
                        para o sistema de refrigeração: 

                        \begin{table}[H]
                            \centering
                            \caption{Testes Iniciais do Sistema de Refrigeração}
                            \label{testes-refrigeracao}
                            \begin{tabular}{|l|l|}
                                \hline
                                Capacidade Efetiva de Chopp & 2,5 Litros \\ \hline
                                Tempo de Aquecimento do Sistema & 29 minutos \\ \hline
                                Tempo de Refrigeração &  20 - 37 minutos\\ \hline
                            \end{tabular}
                        \end{table}
                        
    \section[Sistema de Proteção]{Sistema de Proteção}